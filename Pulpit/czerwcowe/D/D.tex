\documentclass[zad,zawodnik,utf8]{sinol}
\usepackage[utf8]{inputenc}
\usepackage{epigraph}
\title{Czy umiesz potęgować?}
\id{D}
\author{Franciszek Budrowski} % Autor zadania
\pagestyle{fancy}
\iomode{stdin}
\konkurs{Zawody drużynowe}
\etap{ILO Białystok}
\day{}
\date{01.06.2017}
\RAM{32}
\usepackage{mathtools}
\DeclarePairedDelimiter\ceil{\lceil}{\rceil}
\DeclarePairedDelimiter\floor{\lfloor}{\rfloor}

 
\begin{document}
\begin{tasktext}

\iffalse
\epigraph{Man wollte sie zu zwanzig Dingen \\ in einem Haus in Danzig zwingen.}{\textit{Erich Mühsam}}
%Paul Bähre: Danziger Heimatlied (1921)
Danzig sei deutsch!
Danzig, gerissen vom Mutterlande,
Stehst du allein nach der Feinde Gebot.
Danzig, du Perle am Ostseestrande,
Weh klingt deine Klage: Deutschtum in Not!
Deutschtum in Not – Danzig in Not!
Im Staube das Banner schwarz-weiß-rot! 
\fi

	Rafał, z powodu wielu nie cierpiących zwłoki spraw (seriale), nie rozwiązał żadnego zadania z zawodów stałych. Co więcej, termin zgłaszania
rozwiązań mija dzisiaj o 24! Ponieważ Rafał jest bardzo zdolny, potrafi on rozwiązać wszystkie te zadania w ciągu jednego dnia. Jednak kiedy spróbował zalogować się na sprawdzarkę, dowiedział się, że jego hasło zostało zmienione. Gdy powiedział o tym swojemu dobremu przyjacielowi Farałowi, poinformował on Rafała, że to on za tym stoi. Farał postanowił zabawić się kosztem Rafała i przygotował pewną zagadkę, której rozwiązanie doprowadzi Rafała do poznania hasła. Brzmi ona tak:

Stojąc na przeciwko drzwi do sali numer 30 obróć się o 180 stopni i zrób 5 kroków do przodu, następnie znajdź wartości następujących liczb:
	\begin{itemize}
		\item[$\diamond$] $a$ - liczba trofeów po twojej prawej stronie (zliczamy wszystko co znajduje się w gablotce i nie jest pomijalnie małe)
		\item[$\diamond$] $b$ - liczba szyb (każdą powierzchnię przepuszczającą światło nie połączoną bezpośrednio z inną taką powierzchnią zliczamy oddzielnie)
		\item[$\diamond$] $c$ - liczba ławek
		\item[$\diamond$] $d$ - liczba stolików
	\end{itemize}
Oraz oblicz wartość liczby 
 \begin{equation*}
     x = \floor*{\frac{a+b}{3}} + c + d
 \end{equation*}
Następnie udaj się do sali numer $x$ i odwróć się do niej plecami. Kolejna liczba zdefiniowana jest następująco:
	\begin{itemize}
		\item[$\diamond$] $e$ - po lewej znajduje się urządzenie do wchodzenia na górę (lub w dół), $e$ równa się liczbie elementów na które możemy postawić stopy podczas takowej czynności (powierzchnie nie należące do urządzenia pomijamy).
	\end{itemize}
Tajnym kluczem potrzebnym do odzyskania hasła jest liczba $W$ = $a + e$. Pomóż Rafałowi odzyskać kontrolę nad kontem, podczas gdy on będzie rozwiązywał zadania z zawodów stałych.

  	\section{Wejście}
W pierwszej i ostatniej linii wejścia znajduje się liczba $k$ ($0 \leq k \leq 10^{12}$).
	\section{Wyjście}
W pierwszej i ostatniej linii wyjścia należy wypisać resztę z dzielenia liczby $W^{k}$ przez liczbę $37$.
	\makecompactexample
	\section{Punkty cząstkowe}
	W testach wartych $80\%$ punktów $k \leq 10^6$.

\end{tasktext}
\end{document}
