\documentclass[zad,zawodnik,utf8]{sinol}
\usepackage[utf8]{inputenc}
\usepackage{epigraph}
\title{Czy umiesz liczyć?}
\id{B}
\author{Tymon Szworkowski} % Autor zadania
\pagestyle{fancy}
\iomode{stdin}
\konkurs{Zawody drużynowe}
\etap{ILO Białystok}
\day{}
\date{01.06.2017}
\RAM{32}
 
\begin{document}
\begin{tasktext}

Podczas ostatnich zawodów drużynowych Farał dostał nieoczekiwane TLE, dlatego teraz zastanawia się, w jaki sposób dobrze oszacować ilość operacji wykonywanych przez swoje programy. Z racji tego, że rozwiązania Rafała są dość schematyczne, wzory określające liczbę operacji wykonywanych przez każdy z jego programów zbytnio się od siebie nie różnią.

Jeśli rozmiar wczytanego do programu wejścia jest równy $n$, program Rafała wykonuje dokładnie
\begingroup
    \fontsize{13pt}{12pt}\selectfont
	\begin{center}
	$an^3 + bn^2 + cn + d$ 
	\end{center}
\endgroup	
operacji, gdzie $a, b, c, d$ to stałe charakterystyczne dla danego programu Farała. Nasz bohater zastanawia się, jak duże może być wejście aby jego program spełnił wymagania dotyczące liczby operacji narzucone przez organizatorów, które są określone liczbą całkowitą $W$.

Pomóż Rafałowi w znalezieniu największej takiej liczby \textbf{naturalnej} $x$, że
\begingroup
    \fontsize{13pt}{12pt}\selectfont
	\begin{center}
	$ax^3 + bx^2 + cx + d \leq W$ 
	\end{center}
\endgroup	

\section{Wejście}

W pierwszej linii standardowego wejścia znajduje się jedna liczba całkowita $Z$ ($1 \leq Z \leq 10^5$) oznaczająca liczbę programów Farała do rozpatrzenia.  W każdej z kolejnych $Z$ linii znajduje się 5 liczb całkowitych, odpowiednio $a, b, c, d$ ($1 \leq a, b, c, d \leq 5$) oraz $W$ ($1 \leq W \leq 10^{18}$).

\section{Wyjście}
Na standardowe wyjście należy wypisać $Z$ linii. W $i$-tej linii powinna znaleźć się odpowiedź na $i$-te zapytanie Farała będąca największym możliwym rozmiarem wejścia dla którego program Farała (przy założeniu, że Rafał nie zbuguje) przejdzie testy organizatorów. Jeśli dla danego zapytania nie istnieje taka liczba $x$ spełniająca warunki zadania, w odpowiadającej mu linii należy wypisać $chyba$ $nie$.
	\makecompactexample
	
\section{Punkty cząstkowe}
W testach wartych $50\%$ punktów $Z \leq 100$.
\\
\\
\textbf{Uwaga}: Do przechowywania zmiennych polecamy wykorzystanie typu $long$ $long$.

\end{tasktext}
\end{document}
