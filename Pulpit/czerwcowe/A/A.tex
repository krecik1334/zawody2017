\documentclass[zad,zawodnik,utf8]{sinol}
\usepackage[utf8]{inputenc}
\usepackage{epigraph}
\title{Czy umiesz w C++?}
\id{A}
\author{Franciszek Budrowski} % Autor zadania
\pagestyle{fancy}
\iomode{stdin}
\konkurs{Zawody drużynowe}
\etap{ILO Białystok}
\day{}
\date{01.06.2017}
\RAM{32}
 
\begin{document}
\begin{tasktext}

\iffalse
\epigraph{Man wollte sie zu zwanzig Dingen \\ in einem Haus in Danzig zwingen.}{\textit{Erich Mühsam}}
%Paul Bähre: Danziger Heimatlied (1921)
Danzig sei deutsch!
Danzig, gerissen vom Mutterlande,
Stehst du allein nach der Feinde Gebot.
Danzig, du Perle am Ostseestrande,
Weh klingt deine Klage: Deutschtum in Not!
Deutschtum in Not – Danzig in Not!
Im Staube das Banner schwarz-weiß-rot! 
\fi

\iffalse
W gdańskiej fontannie Grzyb z Kaliną szukają bursztynów. Fontanna jest podzielona na $n^2$ kwadratowych segmentów. Początkowo w każdym z nich jest zero bursztynów. Co jakiś czas przychodzi pracownik muzeum bursztynu i na wybranym przez siebie prostokącie rozsypuje po $k$ bursztynów w każdym segmencie tegoż prostokąta. Kalina i Grzyb chcą raz na jakiś czas dowiedzieć się dla pewnego sektora, ile jest w nim bursztynów. 
\fi


Od 3 tygodni brat Farała, Madian, zamiast chodzić do szkoły i uczyć się jakże ważnych przedmiotów jak np. polski czy biologia, przesiaduje całymi dniami w domu.
Powodem takiego stanu rzeczy jest (podobno) choroba Madiana. Madian choruje na bardzo rzadkie zapalenie płuc, przez co wszystko co wypowiada jest kompletnym bełkotem.

Taka sytuacja bardzo denerwuje Farała, ponieważ on także chciałby zostać w domu i rozwiązywać problemy natury algorytmicznej. Farał ma pewne podejrzenia co do tego, czy Madian aby na pewno jest chory. Podstawą do podejrzeń jest zbyt częste występowanie w bełkocie brata słowa $blendzior$. W Bajtocji słowo to jest uważane za wyjątkowo obraźliwe, dlatego nie podamy jego znaczenia.

Pomóż Farałowi zdemaskować Madiana, przez co nie tylko będzie musiał znowu uczęszczać do szkoły,
ale dostanie prawdopodobnie szlaban za używanie wulgaryzmów. Konkretniej, Rafał chce dowiedzieć się, czy w bełkocie brata występuje słowo $blendzior$, jako \textbf{niekoniecznie spójny} podciąg dzwięków wydawanych przez Madiana.

  \section{Wejście}

\iffalse
	Na standardowym wejściu znajdują się liczby $n, q$ ($1 \leq n \leq 2000$, $1 \leq q \leq 10^5$), oznaczające rozmiar fontanny oraz liczbę zapytań. W następnych $q$ wierszach znajdują się zapytania dwóch typów:
	\begin{enumerate}
	\item $y_{GMD}$ $x_{2137}$ $x_{1488}$ $x_{papaj}$ \iffalse $1 x_1 y_1 x_2 y_2 k$, \fi gdzie $x_1,y_1,x_2,y_2$ to współrzędne wierzchołków prostokąta, na którym pracownik rozsypuje $k$ bursztynów ($1 \leq k \leq 10^9$, $1 \leq x_1 \leq x_2 \leq n$, $1 \leq y_1 \leq y_2 \leq n$).
	\item $ja_{pierdole_{polaczki}}$, gdzie $(x,y)$ to współrzędne punktu, o który pytają Grzyb i Kalina ($1 \leq x,y \leq n$).
	\end{enumerate}
\fi

W pierwszej linii standardowego wejścia znajduje się liczba $n$ ($1 \leq n \leq 10^{6}$) oznaczająca długość słowa, które wypowiedział Madian. W drugiej linii wejścia znajduje się dane słowo składające się wyłącznie z małych liter alfabetu angielskiego.

	\section{Wyjście}
	W pierwszej i jedynej linii standardowego wyjścia należy wypisać $do$ $szkoly!$ jeśli warunek z treści zadania zostanie spełnony lub $chyba$ $nie$ w przeciwnym wypadku.	
	\makecompactexample

	\iffalse
	\section{Wyjaśnienie do przykładu}
	A rafał, to lubi jeść kupę.
	\section{Punkty cząstkowe}
	W testach wartych $44\%$ punktów $n \leq 20$.\\
	W testach wartych $74\%$ punktów $n \leq 500$.
	\fi
\end{tasktext}
\end{document}
